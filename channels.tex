\chapter{Channels}\label{channels}

Rob Pike gave a talk at Google about the NewSqueak programming
language, and specifically how it's concurrency features
work. NewSqueak uses channels and is based on the concepts of
Communicating Sequential Processes.

Factor has a 'channels' vocabulary that lets you do similar things.

\wordtable{
\vocabulary{channels}
\ordinaryword{<channel>}{<channel> ( -- channel )}
\ordinaryword{to}{to ( value channel -- )}
\ordinaryword{from}{from ( channel -- value )}
}

The \texttt{<channel>} word creates a channel that threads can send data to and
receive data from. The \texttt{to} word sends data to a channel. It is a
synchronous send and blocks waiting for a receiver if there is
none. The \texttt{from} word receives data from a channel. It will block if
there is no sender.

There can be multiple senders waiting, and if a thread then receives
on the channel, a random sender will be released to send its
data. There can also be multiple receivers blocking. A random one is
selected to receive the data when a sender becomes available.

Here is the `counter' example ported from NewSqueak:

\begin{verbatim}
USING: channels threads ;

: (counter) ( channel n -- )
  [ swap to ] [ 1 + (counter) ] 2bi ;
    
: counter ( channel -- )
  2 (counter) ;    

: counter-test ( -- n1 n2 n3 )
  <channel> dup [ counter ] curry "counter" spawn drop
  [ from ] [ from ] [ from ] tri ; 

counter-test . . . 
 => 2
    3
    4
\end{verbatim}

Given a channel, the \texttt{counter} word will send numbers to it,
incrementing them by one, starting from two. \texttt{counter-test} creates a
channel, spawns a process to run \texttt{counter}, and then receives a few
values from the channel.

