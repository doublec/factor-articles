\chapter{Lazy}\label{lazy}

I originally wrote the Lazy Lists library in Factor to support the
Parser Combinators library. It was one of my first libraries in Factor
and was probably not very well written. Matthew Willis worked on it
for the recent Factor releases to get it working with the removal of
the standard list datatype and made it a lot better in places.

Both the lazy lists and parser combinators worked by building up
quotations manually to defer computation, and having these quotations
called when required. This unfortunately made the library a bit
difficult to debug and understand. While using the parser combinators
library for a recent project I found it quite hard to work some things
out and found a couple of bugs in the parser combinators - and areas
where the lazy lists weren't lazy enough.

To resolve the issues I rewrote the lazy lists library in a slightly
different style and plan to redo the parser combinators in a similar
manner. Instead of building up quotations I now use tuples to hold the
information, with generic functions to call to process them.

I created a generic function based protocol for lists. It has three
generic functions:

\begin{verbatim}
GENERIC: car  ( cons -- car )
GENERIC: cdr  ( cons -- cdr )
GENERIC: nil? ( cons -- bool )
\end{verbatim}


A 'cons' tuple for normal non-lazy lists, with the obvious
implementation was the starting point:

\begin{verbatim}
TUPLE: cons car cdr ;
M: cons car ( cons -- car )
    cons-car ;    

M: cons cdr ( cons -- cdr )
    cons-cdr ;    

: nil ( -- cons )
  T{ cons f f f } ;

M: cons nil? ( cons -- bool )
    nil eq? ;
\end{verbatim}


This gives the functionality of ordinary lists that used to exist in Factor. To make actual lazy lists I use the 'force' and <promise> words from Matthew's lazy list rewrite. A <promise> is a wrapper around a quotation that calls that quotation when 'force' is called on it, and memoizes the value to return directly on later 'force' calls. A lazy list has a promise for both the 'car' and 'cdr':

\begin{verbatim}
: lazy-cons ( car cdr -- promise ) 
    >r <promise> r> <promise> <cons> 
    T{ promise f f t f } clone [ set-promise-value ] keep ;

M: promise car ( promise -- car )
  force car force ;

M: promise cdr ( promise -- cdr )
  force cdr force ;

M: promise nil? ( cons -- bool )
  force nil? ;
\end{verbatim}


Notice that the lazy list itself is a promise. And the methods are
specialized on the <promise> type. So not only are the 'car' and 'cdr'
promises, but so is the list itself. The 'cdr' of a lazy list is a
promise, which when forced, returns something that conforms to the
list protocol. This means parts of the list can be lazy, and parts
non-lazy

A lazy map operation is the first word I implemented on top of this
generic protocol. Previous implementations used quotations that were
automatically generated and were quite complicated. In this new
implementation I created a <lazy-map> tuple that implemented the list
protocol which turned out to be much simpler:

\begin{verbatim}
TUPLE: lazy-map cons quot ;

: lmap ( list quot -- result )
    over nil? [ 2drop nil ] [ <lazy-map> ] if ;

M: lazy-map car ( lazy-map -- car )
  [ lazy-map-cons car ] keep
  lazy-map-quot call ;

M: lazy-map cdr ( lazy-map -- cdr )
  [ lazy-map-cons cdr ] keep
  lazy-map-quot lmap ;

M: lazy-map nil? ( lazy-map -- bool )
  lazy-map-cons nil? ;
\end{verbatim}


Basically the 'lmap' call itself just returns a <lazy-map> which contains the quotation and list to map over. Calls to 'car' will call the quotation on the head of the list, while 'car' returns a new <lazy-map> that holds the same quotation and the 'cdr' of the original list.

The 'take' operation returns the first 'n' items in the list (whether
lazy or not). It's lazy implementation is:

\begin{verbatim}
TUPLE: lazy-take n cons ;

: ltake ( n list -- result )
    over zero? [ 2drop nil ] [ <lazy-take> ] if ;
     
M: lazy-take car ( lazy-take -- car )
  lazy-take-cons car ;

M: lazy-take cdr ( lazy-take -- cdr )
  [ lazy-take-n 1- ] keep
  lazy-take-cons cdr ltake ;

M: lazy-take nil? ( lazy-take -- bool )
  lazy-take-n zero? ;
\end{verbatim}


Given a word 'naturals' to return an infinite list of natural numbers,
 the squares of the first 10 numbers can be returned as:

\begin{verbatim}
naturals [ dup * ] lmap 10 swap ltake
\end{verbatim}


The result is a lazy list itself. No actual computation is done yet
until 'car' or 'cdr' is called on the result.

More code implements 'subset' to filter out items from the lists,
append lazy lists in a lazy manner, etc. The squares of all odd
natural numbers are expressed as:

\begin{verbatim}
naturals [ odd? ] lsubset [ dup * ] lmap
\end{verbatim}


It would be possible to add to this to implement the protocol for
other things like lines in a file. This way you can lazily read
through a large file line by line. Although this could be done in the
previous lists code I think this approach makes things easier to
read. As part of the refactoring I also wrote integrated Factor help
for all the non-internal words.

I'm not sure how 'concatenative' this approach is or whether it's the
right approach for 'stack' based languages but it seems to work well
for Factor. I'll work on the parser combinators, fitting them into a
similar style, and see how it works out with the current project.


