\chapter{Cells}\label{cells}

Factor has a Cells like mechanism for propagating changes in models to
their GUI representation (called Gadgets). It's like a lightweight
functional reactive programming system. 

\section{Models}
To have a gadget dynamically updated it must reference a `model' which
wraps the value to be displayed. Whenever the data wrapped by the
model is changed, a sequence of connected objects are notified of the
change.  

To create a model, use the \verb|<model>| word, with the value it is wrapping
on the top of the stack.

\wordtable{
\vocabulary{models}
\ordinaryword{<model>}{<model>~( value -- model )}
}
\begin{alltt}
USE: models
42 <model> value>> .
  \textbf{42}
\end{alltt}

New models can be created based on the value of existing models. The \verb|<arrow>| word creates a new model, with its value being the result of a quotation applied to the original models value. Its value changes whenever the original model is changed.
\wordtable{
\vocabulary{models.arrow}
\ordinaryword{<arrow>}{<arrow>~( model quot -- arrow )}
}
\begin{alltt}
USE: models
USE: models.arrow

SYMBOL: a
SYMBOL: b

42 <model> a set
a get [ 2 * ] <arrow> b set
b get activate-model
b get value>> .
  \textbf{84}
10 a get set-model
b get value>> .
  \textbf{20}
\end{alltt}
You'll notice the call to the word \verb|activate-model|. An active model is one that gets updated when the models it depends on change. There is a corresponding \verb|deactivate-model| that can be called to stop changes from propagating.

\wordtable{
\vocabulary{models}
\ordinaryword{activate-model}{activate-model~( model -- )}
}

\wordtable{
\vocabulary{models}
\ordinaryword{deactivate-model}{deactivate-model~( model -- )}
}

\section{Gadgets and Models}
Gadgets can use models as their underlying data to be displayed. The gadget will automatically be updated when the underlying data changes. 

\begin{verbatim}
USE: ui.gadgets.labels
USE: ui.gadgets.panes

"hello"  <model> 
dup <label-control> gadget.
"world" over set-model
"!" over set-model
\end{verbatim}

Here we create a model containing the text ``hello''. A label-control gadget is created with this model as the data, and the gadget is printed in the listener with \verb|gadget.|. The label will show ``hello'' as the text. When the model is updated with new text using \verb|set-model|, the label control displays the new text immediately.

\section{Updating time example}

In this example I create a global value called `time' that holds a
model wrapping the current date and time:

\begin{verbatim}
USING: calendar models ;


SYMBOL: time
now <model> time set-global
\end{verbatim}

The model here wraps the current date and time produced by
\verb|now|. To have the value change regularly I start a background
thread which sets the value every second:

\begin{verbatim}
USE: timers

: update-time ( -- )
  now time get-global set-model ;

[ update-time ] 1 seconds every drop
\end{verbatim}

The \verb|update-time| words gets the current date and time (using \verb|now|)
and sets the model's value to it. A timer is created to do this every second.

We can confirm that this value is updating by getting the value of
the `time' variable and see that it changes between calls:

\begin{alltt}
USE: calendar.format

time get-global value>> timestamp>http-string .
  \textbf{"Sun, 20 Jan 2008, 02:02:02 GMT"} 

time get-global value>> timestamp>http-string .
  \textbf{"Sun, 20 Jan 2008, 02:04:02 GMT" }
\end{alltt}

A label gadget that displays this value, and is updated as it changes,
can be created and displayed with:

\begin{verbatim}
USING: ui.gadgets.labels ui.gadgets.panes calendar.format ;

time get-global 
[ timestamp>http-string ] <arrow> <label-control> 
gadget.
\end{verbatim}

Notice I'm using \verb|<arrow>| to take the time value and convert it into a string. A \verb|<label-control>| can only display strings, so it has the filtered value as its data. The gadget displayed on the listener will update automatically as the time changes every second.

