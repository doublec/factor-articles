\chapter{Web Applications}\label{webapps}

There are a number of different ways of writing web applications in
Factor but for this approach I'm using the furnace framework.

The first step is to start the web server. This lives in the vocab \texttt{http.server}:
\begin{verbatim}
USE: http.server
[ 8888 httpd ] in-thread
\end{verbatim}

This will start an instance of the server on port 8888 in another
thread, to allow us to continue to enter commands in the listener.

By default web applications are accessed on the URL path
\texttt{/responder/name}, where \texttt{name} is the name of the web application.

Accessing the web application path runs an 'action'. An action
produces HTML output which gets sent back to the client browser. A web
application has a default 'action' that gets run (the equivalent of an
index.html), and can have other actions that are specified in the
URL. Some examples:
\begin{verbatim}
http://localhost:8888/responder/foo
    Runs the default action for the 'foo' web application
http://localhost:8888/responder/foo/doit
    Runs the 'doit' action
http://localhost:8888/responder/foo/hello?name=chris
    Runs the 'hello' action giving the argument 'name' with the value 'chris'
\end{verbatim}

The syntax for furnace URL's is therefore
\texttt{http://servername:port/responder/[webappname]/[action]?[arguments]}

Furnace web application must exist under the \texttt{webapps} vocabulary. So
accessing \texttt{/responder/foo} will look for furnace details in the
vocabulary \texttt{webapps.foo}.

A furnace web application is registered with the http server using the
\verb|web-app| word. It takes three arguments on the stack.

\wordtable{
\vocabulary{furnace}
\ordinaryword{web-app}{web-app ( name default path -- )}
}

The \texttt{name} is the vocabulary name of the web application with
out the \texttt{webapps} prefix. \texttt{default} is the name of the
action that gets run when the web application URL is
accessed. \texttt{path} is the location of any template files the web
application uses.

An action is a word that outputs data to be sent to the browser. It
can be as simple as:
\begin{verbatim}
: doit ( -- ) serving-text "I am here" print ;
\end{verbatim}

The word must be registered as an action:

\begin{verbatim}
\ doit { } define-action
\end{verbatim}

Now accessing the URL for the web application with \texttt{doit} at
the end of the path will result in 'I am here' being sent to the
browser. Note the \texttt{serving-text} call. That outputs the headers
for the mime type and the standard HTTP response. There is also a
\texttt{serving-html}, or you could write the headers manually.

Actions can take arguments. These are placed on the stack for the word
that is called:
\begin{verbatim}
: hello ( name -- ) serving-text "Hello " write print ;
\ hello { { "hello" } } define-action
\end{verbatim}

So the complete code for the simplest of web applications is:

\begin{verbatim}
USE: http.server
[ 8888 httpd ] in-thread
IN: webapps.test
USE: furnace

: index serving-text "We're alive!" print ;
\ index { } define-action 

: hello ( name -- ) serving-text "Hello " write print ;
\ hello { { "name" } } define-action

"test" "index" "." web-app
\end{verbatim}

Accessing \texttt{http://localhost:8888/responder/test} will run the 'index'
action. This is what we passed as the 'default' parameter on the stack
to the 'web-app' word. Accessing
\texttt{http://localhost:8888/responder/test/hello?name=chris} will run the
'hello' action.

There is also the facility to have template files, very much like
JSP. The 'path' parameter to \texttt{web-app} defines the location of
these. Inside your action word you can call 'render-template' to run
the template and have it sent to the browser:
\begin{verbatim}
: runme ( -- ) f "page" "Title" render-template ;
\ runme { } define-action
\end{verbatim}

This will load the 'page.furnace' file in the path given to 'web-app'. It should contain standard HTML with embedded Factor code inside <% and %> tags. It will be run and sent to the client. The 'f' passed in this example can be an instance of a tuple (an object) and the template can access the slots of that instance to display data, etc.

There is quite a bit more that can be done. There is a continuation
based workflow system, validators for actions, etc. There is also much
more that needs to be done. handling sessions, cookies, etc. Hopefully
this post gives a quick introduction and allows you to get started.

