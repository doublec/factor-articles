\chapter{Git Repository}\label{git}

\section{How to publish a git repository}

To set up a repository on a server you should clone the existing
Factor repository using the `--bare' option:

\begin{verbatim}
git clone --bare http://www.factorcode.org/git/factor.git factor.git
\end{verbatim}

A bare repository is one without a checked out working copy of the
code. It only contains the git database. As a general rule you should
never push into a repository that contains changes in the working
copy. To ensure this doesn't happen, we're making the server
repository a bare repository - it has no working copy.

Copy the `factor.git' directory onto your server. I put it in
`/git/factor.git'. Now if you have changes on your local machine that
you want to push to your repository you can use something like:

\begin{verbatim}
git push yourname@yourserver.com:/git/factor.git
\end{verbatim}

If you want to push changes from a specific branch in your local
repository:

\begin{verbatim}
git push yourname@yourserver.com:/git/factor.git mybranch:master
\end{verbatim}

To publish the remote repository you have two options. You can publish
via the HTTP protocol, or via the git protocol. The first is slower
but usable by people behind restrictive firewalls, while the second is
more efficient but requires an open port. I suggest doing both.

To publish via HTTP, you must make the file `hooks/post-update'
executable:

\begin{verbatim}
chmod +x /git/factor.git/hooks/post-update
\end{verbatim}

This gets executed whenever something is pushed to the repository. It
runs a command `git-update-server-info' which updates some files that
makes the HTTP retrieval work. You should also run this once manually:

\begin{verbatim}
cd /git/factor.git
git-update-server-info
\end{verbatim}

Now make the /git directory published via your webserver (I symbolic
link to it in the server's doc-root). People can pull from the
repository with:

\begin{verbatim}
git pull http://yourserver.com/git/factor.git
\end{verbatim}

To set up the git protocol you need to run the `git-daemon'
command. You pass it a directory which is the root of your git
repositories. It will make public all git repositories underneath that
root that have the file `git-daemon-export-ok' in it. So first create
this file:

\begin{verbatim}
touch /git/factor.git/git-daemon-export-ok
\end{verbatim}

Run the daemon with:

\begin{verbatim}
git-daemon --verbose /git
\end{verbatim}

The `--verbose' will give you output showing the results of connecting
to it. I run this from within a screen session. You can set it up to
automatically run using whatever features your server OS has. Now
people can retrieve via the git protocol:

\begin{verbatim}
git pull git://yourserver.com/git/factor.git
\end{verbatim}

My repository is accessible from both protocols:

\begin{verbatim}
git clone http://www.double.co.nz/git/factor.git
git clone git://double.co.nz/git/factor.git
\end{verbatim}

\section{Binary Files}

Git has a heuristic for detecting binary files. You can force other
file types to be binary by adding a .gitattributes file to your
repository. This file contains a list of glob patterns, followed by
attributes to be applied to files matching those patterns. By adding
.gitattributes to the repository all cloned repositories will pick
this up as well.

For example, if you want all *.foo files to be treated as binary files
you can have this line in .gitattributes:

\begin{verbatim}
*.foo -crlf -diff -merge
\end{verbatim}

This will mean all files with a .foo extension will not have carriage
return/line feed translations done, won't be diffed and merges will
result in conflicts leaving the original file untouched.

Now when you pull from another repository that has changes to a .foo
file you'll see something like:

\begin{verbatim}
 test.foo |  Bin 32 -> 36 bytes
\end{verbatim}

Note that it shows it is a binary file. If you pull from another
repository with changes to test.foo you'll get:

\begin{verbatim}
Auto-merged test.foo
CONFLICT (content): Merge conflict in test.foo
\end{verbatim}

The file will be untouched and you can change it manually to be the
correct version. Either by leaving it untouched, or copying a new file
over it. Then you need to commit the merge conflict fix (even if you
left the file untouched):

\begin{verbatim}
git commit -a -m "Fix merge conflict in test.foo"
\end{verbatim}

\section{Cherry Picking}

The cherry picking of patches works differently to Darcs. There are a
couple of ways of handling this, but I use `git cherry-pick'. If you
have a number of contributers with their own repositories that you
regularly pull from you can set up remote tracking branches:

\begin{verbatim}
git remote add john http://...
git remote add mary http://...
\end{verbatim}

Now when you want John and Mary's most recent patches you can fetch
them:

\begin{verbatim}
git fetch john
git fetch mary
\end{verbatim}

This does not make any changes to your local branches. It gets and
stores their changes in a separate remote tracking branch. If you want
to see what John has changed, compared to yours:

\begin{verbatim}
git log -p master..john/master
\end{verbatim}

From there you can decide to pull in all John's commits:

\begin{verbatim}
git merge john/master
\end{verbatim}

If you want one commit, but not its dependencies then this is where
`cherry-pick' is used.

Given a commit id, `cherry-pick' will take the patch for that commit
and apply it to your current branch. It's used like:

\begin{verbatim}
git cherry-pick abcdefgh
\end{verbatim}

This creates a commit with a different commit id than the original,
but with the same contents. It needs to be a different id as it
doesn't have the same dependencies as the original.

If you decide later you want all John's commits and do a merge which
includes the commit that you cherry picked from you might expect
conflicts. Git handles this case fine and does an automatic merge,
noticing the patches are the same. So it effectively gives you the
same functionality as Darcs selective patch pulling, but not as nice a
user interface.
