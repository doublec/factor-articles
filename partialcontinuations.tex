\chapter{Partial Continuations}\label{partialcontinuations}

I've written some partial continuation support and put it in the
Factor contrib directory. It implements bshift and breset as outlined
by Oleg's post on comp.lang.scheme.


It should be relatively easy to convert Oleg's Scheme examples to
Factor. Just remember that the partial continuation has stack effect (
a -- b ) and the quotations passed to bshift and breset have stack
effect ( pcc -- v ) and ( -- v ) respectively.

`breset' marks the scope of the partial continuation. If `bshift' is
not used then the value returned by the quotation is left on the
stack:

\begin{verbatim}
[ drop 5 ] breset
 => 5
\end{verbatim}


An example with `bshift':

\begin{verbatim}
[
 1 swap [ 5 swap call ] bshift +
] breset
 => 6
\end{verbatim}


In this case the partial continuation passed to the `bshift' quotation
represents the computation `1 X +' where `X' is replaced by the value
passed to the partial continuation. In this case 5, resulting in a
result of 6. Additional calls can be made to the same continuation:

\begin{verbatim}
[
 1 swap [ 5 over call swap call ] bshift +
] breset
 => 7
\end{verbatim}


This calls the `1 X +' partial continuation twice. First with `5'
returning the value `6'. Which is then passed to it again, returning
`7'. The partial continuation does not need to be called. Values that
`fall through' cause the result to be returned from the `breset'
quotation:

\begin{verbatim}
[
 1 swap [ drop 5 ] bshift +
] breset
 => 5
\end{verbatim}


Here's a fun example translated from Oleg's posting. The following
`range' function has some interesting properties:

\begin{verbatim}
 : range ( r from to -- )
 rot [ ( from to pcc -- )
   -rot [ over + pick call drop ] each 2drop f
 ]  bshift ;
\end{verbatim}


It uses the standard `each' call on a `from' and `too' number to call
a quotation on each number between `from' and `too' inclusive. The
quotation called is the partial continuation provided by
`bshift'. This can be used in code like:

\begin{verbatim}
[ 1 5 range . ] breset drop
 => 1
    2
    3
    4
    5
\end{verbatim}


For each item in the range it executed the partial continuation which
is `X .', printing that item in the range. So given a function that
knows nothing about the special capability of `range' can still work
with it. The following prints the first five factorials:

\begin{verbatim}
: fact ( n -- n ) dup 1 = [ 1 ] [ dup 1 - fact ] if * ;

[ 1 5 range fact . ] breset drop
 => 1
   2
   6
   24
   120
\end{verbatim}


I'm not sure how useful delimited continuations are in Factor but it
 gives something to play with to see how they work.

I've made some changes to the parser combinator library I wrote for
 Factor. The changes were to make the usage of the library to be more
 consistent with good Factor style.

The first change was to ensure that items on the stack are accessable
 in an intuitive manner from within the quotations passed to breset
 and bshift. When writing combinators in Factor it's important that
 the internals of the combinator implementation do not affect the
 callers view of items on the stack. For example:

\begin{verbatim}
20 [ drop 2 * ] breset
\end{verbatim}


Recall that the quotation passed to breset has stack effect ( r -- v
). Once the `r' is dropped the `*' should be expected to operate on
the `2' and the `20'. The `20' is accessable as if it was directly
under the `r' on the stack from within the quotation.

In a prior implementation of breset I called the quotation using
`call' after creating a `dup' of the `r':


\begin{verbatim}
: breset 
  ... ( quot r -- )
  dup rot call ( r v -- )
\end{verbatim}


The problem with this is that from within the quotation there is an
 extra `r' above items on the stack before the quotation is called:

\begin{verbatim}
20 [ drop 2 * ] breset
\end{verbatim}


The stack on entry to the quotation here is ( 20 r r -- ). Changing
the breset implementation to use `keep' instead of call fixes this
problem:


\begin{verbatim}
: breset 
  ... ( quot r -- )
  swap keep ( v r -- )
\end{verbatim}


Notice also that the return item from the quotation, `v', is now below
the `r' and I didn't need to `dup' it. In general, usage of words like
`keep' will enable combinators to more intuitively access stack items
from outside the quotation.

Another way of solving this problem is to use the return stack words
`>r' and `r>' to move items temporarily off the stack so that the
quotation being called is imediately above the relevant stack items
supplied by the caller.

The second major change was to remove the use of `with-scope' and
namespaces to store the continuation delimiter. I previously stored
this in a `mark' and `mark-old' variable. Now the implementations of
breset and bshift manage these on the stack.

Storing variables in namespaces is seductive. It enables you to avoid
sometimes complicated stack management by giving you named
variables. Unfortunately it comes with a price. When setting the value
of a variable in a namespace it is stored in the top most namespace on
the namespace stack. This can be seen with code like this:


\begin{verbatim}
SYMBOL: myvar
5 myvar set
myvar get .
 => 5
10 [ myvar set ] keep drop myvar get .
 => 10
\end{verbatim}


As expected setting the myvar variable in the quotation passed to
 `keep' results in the global myvar value being set to 10. But if this
 is run inside a `with-scope' you get caught by the fact that a new
 namespace is at the top of the stack:


\begin{verbatim}
myvar get .
  => 10
20 [ [ myvar set ] with-scope ] keep drop myvar get .
  => 10
\end{verbatim}


Notice the value is still 10. This is because the variable is set in
the new namespace which is dropped off the namespace stack when the
`with-scope' exits. Normally you would be aware of this when you write
code, but if you use a combinator that uses `with-scope' in its
implementation, and it then calls your quotation from within that
scope then all your variable sets will be lost at the end of the
combinator call.

This may be desired, and the reason why the combinator is in a scope,
but for many cases it's not the desired behaviour. So as a general
rule I try to avoid using variables and with-scope in combinator
implementations.

There was also a minor bug in my `range' example in that it didn't
give the correct range if the starting number was anything but
`1'. The corrected `range' implementation is:


\begin{verbatim}
: range ( r from to -- n )
 over - 1 + rot [ 
   -rot [ over + pick call drop ] each 2drop f  
 ]  bshift 2nip ;
\end{verbatim}


Note the `2nip' at the end. `bshift' and `breset' operate like other
continuation combinators in that they restore the stack to what they
were before they were called. In this case the `from' and `to' were on
the stack and we need to `nip' them off. Simple usage works as before:


\begin{verbatim}
[ 1 5 range . f ] breset drop
 => 1
    2
    3
    4
    5
\end{verbatim}


Nested usage works correctly now:


\begin{verbatim}
[ dup 1 3 range swap 10 12 range + . f ] breset drop
 => 11
    12
    13
    12
    13
    14
    13
    14
    15
\end{verbatim}


It can be hard to reason about what usage of words like `range'
does. Think of it as returning a single value `n', and calling the
breset multiple times with the value of `n' for each `n' in the
range. So you'll see that after the first `range' call I `swap' to
swap the result of the range and get the continuation mark passed to
breset back to the top of the stack to call the second `range'.

You'll notice that the result of calling these snippets always returns
`f' on the stack. This is because the range implementation leaves `f'
on the stack at the end of the bshift quotation. This results in
`breset' exiting with that value.

A question was raised on the Factor mailing list about CLU-style
iterators. An example of the usage from one of the links in that
message is:


\begin{verbatim}
sum: INT := 0;
loop sum := sum + 1.upto!(10); end;
#OUT + sum + '\n'; -- Prints sum of integers from 1 to 10
\end{verbatim}


This has a very direct translation in Factor using range and breset
as:


\begin{verbatim}
SYMBOL: sum
0 sum set 
[ 1 10 range sum get + sum set f ] breset drop
sum get .
 => 55
\end{verbatim}

